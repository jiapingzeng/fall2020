\documentclass{article}
\usepackage[margin=1in]{geometry}
\usepackage{enumitem}
\usepackage{setspace}
\usepackage{amsmath}
\usepackage{amssymb}
\usepackage{physics}
\usepackage{relsize}

\title{Math 180 Homework 5}
\date{11/20/2020}
\author{Jiaping Zeng}

\begin{document}
\setstretch{1.35}
\maketitle

\begin{itemize}
      \item [8.3.1]
            \textbf{Answer}: Each component of such a graph is a directed cycle since by construction, each vertex has exactly one edge going in and one edge going out, therefore forming a directed cycle by definition. Since our original graph is a tree, it does not contain any cycle, therefore each component will have only the constructed directed cycle and no other cycles. Then, the branches attached to the chord in the original graph are simply attached onto the cycles, with the edges directed towards the cycle by construction.
      \item [8.3.2]
            \textbf{Answer}: We can reconstruct the chord first by looking at the vertices in the cycles; we can first order the vertices by their labels, i.e. $v_i,v_{i+1},v_{i+2},\ldots$, then construct a new list by finding the vertex each vertex in our ordered list is connected to. This gives us the chord. Then for the branches, we can simply take them from the cycle and reattach to the chord, and make them undirected.
      \item [8.4.1]
            \textbf{Answer}: Since $T$ is a tree, we have $\sum_im_i=n-2$. In addition, we also have $\abs{E}=n-1$, so $\sum_i(\deg_T(i)-1)=2\abs{E}-n=2(n-1)-n=n-2$. Then $\sum_im_i=\sum_i(\deg_T(i)-1)$, so we must have $m_i=\deg_T(i)-1\implies\deg_T(i)=m_i+1$ for all $i$.
      \item [P10] Use the Prüfer code to count the number of spanning trees of $K_n$ with $\deg(v_1)=2$.\\
            \textbf{Answer}: Since $\deg(v_1)=2$, we have $m_1=1$ by problem 8.4.1, i.e. $v_1$ shows up only once in the Prüfer code. Then since $K_{n-1}$ has $(n-1)^{n-3}$ spanning trees by Cayley's formula, it also has $(n-1)^{n-3}$ unique Prüfer codes. We can place $v_1$ anywhere in the Prüfer codes (each has length $n-2$), so there are $(n-2)(n-1)^{n-3}$ such spanning trees.
      \item [P11] Use the Prüfer code to count the number of spanning trees of $K_n$ with each vertex degree 1 or 3.\\
            \textbf{Answer}: By problem 8.4.1, spanning trees have $\deg_T(i)=m_i+1$ for all $i$, so our spanning trees must have $m_i=0$ or $m_i=2$. In other words, each vertex in the code shows up exactly twice. Then, since $\sum_im_i=n-2$, exactly $\frac{n-2}{2}$ vertices are in the code. Then we have $(n-2)!$ ways to arrange the vertices, with $2^\frac{n-2}{2}$ duplicates for each code. Therefore we have $\mathlarger{\binom{n}{\frac{n-2}{2}}\cdot\frac{(n-2)!}{2^\frac{n-2}{2}}}$ spanning trees in $K_n$.
\end{itemize}
\end{document}
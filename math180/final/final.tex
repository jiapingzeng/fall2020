\documentclass{article}
\usepackage[margin=1in]{geometry}
\usepackage{setspace}
\usepackage{amsmath}
\usepackage{amssymb}
\usepackage{physics}
\usepackage{graphicx}
\usepackage{relsize}
\usepackage{graphicx}

\title{Math 180 Midterm 2}
\author{Jiaping Zeng}
\date{11/20/2020}

\begin{document}
\setstretch{1.5}

\newpage
I certify on my honor that I have neither given nor received any help, or used any non-permitted resources, while completing this evaluation.\\
Signature:\includegraphics[width=2in]{signature.png}\\
Date: 12/14/2020

\newpage
\begin{itemize}
      \item [Q2]
            \begin{itemize}
                  \item [(a)]
                  \item [(b)]
                  \item [(c)]
                  \item [(d)]
                  \item [(e)]
                  \item [(f)]
                  \item [(g)]
                  \item [(h)]
            \end{itemize}
\end{itemize}

\newpage
\begin{itemize}
      \item [Q3]
            \begin{itemize}
                  \item [(a)] Prove that $r(k,2)=k$ for all $k\geq 2$.\\
                        \textbf{Answer}: Since we have $r(k,\ell)=1+\max\{\abs{V(G)}:\omega(G)<k\text{ and }\alpha(G)<\ell\}$, we can start by finding $\abs{V(G)}$ that satisfies both $\omega(G)<k$ and $\alpha(G)<2$. Since $\alpha(G)<2$, $G$ has to be a complete graph, or we else can take two vertices that are not connected to have $\alpha(G)\geq 2$. Then to have $\omega(G)<k$ with $G$ begin a complete graph, we can only have up to $k-1$ vertices. Therefore $\max\{\abs{V(G)}:\omega(G)<k\text{ and }\alpha(G)<\ell\}=k-1$ and $r(k,2)=1+\max\{\abs{V(G)}:\omega(G)<k\text{ and }\alpha(G)<2\}=k$.
                  \item [(b)] Prove that $r(k,\ell)\leq r(k-1,\ell)+r(k,\ell-1)$ for all $k,\ell\geq 2$.\\
                  \textbf{Answer}:
                  \item [(c)] Use parts (a) and (b) above to obtain an upper bound for $r(5,5)$.\\
                  \textbf{Answer}: By part (a) we have $r(k,2)=k$ and by symmetry we also have $r(2,\ell)=\ell$. Then using part (b), We have \\
                  $r(5,5)$\\$\leq r(4,5)+r(5,4)$\\$\leq (r(3,5)+r(4,4))+(r(4,4)+r(5,3))$\\$=r(3,5)+2r(4,4)+r(5,3)$\\$\leq r(2,5)+r(3,4)+2(r(3,4)+r(4,3))+r(4,3)+r(5,2)$\\$=r(2,5)+3r(3,4)+3r(4,3)+r(5,2)$\\$\leq 5+3(r(2,4)+r(3,3))+3(r(3,3)+r(4,2))+5$\\$=10+3r(2,4)+6r(3,3)+3r(4,2)$\\$\leq 10+3\cdot 4+6(r(2,3)+r(3,2))+3\cdot 4$\\$=34+6\cdot 6$\\$=70$.
            \end{itemize}
\end{itemize}

\newpage
\begin{itemize}
      \item [Q4]
            \begin{itemize}
                  \item [(a)]
                  \item [(b)]
                  \item [(c)]
                  \item [(d)]
            \end{itemize}
\end{itemize}

\newpage
\begin{itemize}
      \item [Q5]
            \begin{itemize}
                  \item [(a)]
                  \item [(b)]
                  \item [(c)]
            \end{itemize}
\end{itemize}

\end{document}
\documentclass{article}
\usepackage[margin=1in]{geometry}
\usepackage{setspace}
\usepackage{amsmath}
\usepackage{amssymb}
\usepackage{physics}
\usepackage{graphicx}
\usepackage{relsize}
\usepackage{graphicx}

\title{Math 180 Midterm 2}
\author{Jiaping Zeng}
\date{11/20/2020}

\begin{document}
\setstretch{1.35}

\newpage
I certify on my honor that I have neither given nor received any help, or used any non-permitted resources, while completing this evaluation.\\
Signature:\includegraphics[width=2in]{signature.png}\\
Date: 12/14/2020

\newpage
\begin{itemize}
      \item [Q2]
            \begin{itemize}
                  \item [(a)] Any graph on 3 or more vertices with no triangle is bipartite.\\
                        \textbf{Answer}: False; take any cycle $C_n$ with odd n, which is triangle-free yet not 2-colorable and therefore not bipartite.
                  \item [(b)] If $(d_1,\ldots,d_n)$ is the score of a graph with at least 1 edge, then $(d_1,\ldots,d_n,2)$ is a valid graph score.\\
                        \textbf{Answer}: True; we can remove an edge and connect a new vertex to the vertices connected by the removed edge.
                  \item [(c)] Any connected, planar graph is 2-connected.\\
                        \textbf{Answer}: False; take any path $P_n$ which is connected but not 2-connected.
                  \item [(d)] If $T$ is a spanning tree of $K_n$, then the complement $T^c$ is not a spanning tree of $K_n$ unless $n=1$ or $n=4$.\\
                        \textbf{Answer}: True; suppose $T$ and $T^c$ are both spanning trees of $K_n$, then they must each have $n-1$ edges. So $K_n$ must has $2n-2$ edges by adding up the edges from $T$ and $T_c$. Since $K_n$ have $\frac{n(n+1)}{2}$ edge by double counting, we can see that we only have $2n-2=\frac{n(n-1)}{2}$ when $n=1$ or $n=4$.
                  \item [(e)] If $G$ has more than 5 vertices and is triangle free, then $G^c$ cannot be triangle-free.\\
                        \textbf{Answer}: By Ramsey's theorem we have $\omega(G)\geq 3$ or $\alpha(G)\geq 3$ for $\abs{V(G)}\geq 6$. Since $G$ is triangle-free we must have $\alpha(G)\geq 3$, but those three independent vertices are connected in $G^c$ which forms a triangle.
                  \item [(f)] Every bipartite graph $G$ has chromatic number $\chi(G)=2$.\\
                        \textbf{Answer}: False; take any empty bipartite graph which has $\chi(G)=1$.
                  \item [(g)] If $G$ and $H$ are distinct Eulerian graphs on the same vertex set, then their symmetric difference (of their edge set) is Eulerian.\\
                        \textbf{Answer}: False; fix one vertex, take two Eulerian graphs that overlap on edges connected to that vertex. Then the symmetric difference is disconnected and therefore cannot be Eulerian.
                  \item [(h)] Distinct spanning trees of a graph $G$ give distinct bases for the cycle space of $G$.\\
                        \textbf{Answer}: False; take $K_3$ which has basis $\{C_3\}$, but there are three distinct possible spanning trees by rotation.
            \end{itemize}
\end{itemize}

\newpage
\begin{itemize}
      \item [Q3]
            \begin{itemize}
                  \item [(a)] Prove that $r(k,2)=k$ for all $k\geq 2$.\\
                        \textbf{Answer}: Since we have $r(k,\ell)=1+\max\{\abs{V(G)}:\omega(G)<k\text{ and }\alpha(G)<\ell\}$, we can start by finding $\abs{V(G)}$ that satisfies both $\omega(G)<k$ and $\alpha(G)<2$. Since $\alpha(G)<2$, $G$ has to be a complete graph, or we else can take two vertices that are not connected to have $\alpha(G)\geq 2$. Then to have $\omega(G)<k$ with $G$ begin a complete graph, we can only have up to $k-1$ vertices. Therefore $\max\{\abs{V(G)}:\omega(G)<k\text{ and }\alpha(G)<\ell\}=k-1$ and $r(k,2)=1+\max\{\abs{V(G)}:\omega(G)<k\text{ and }\alpha(G)<2\}=k$.
                  \item [(b)] Prove that $r(k,\ell)\leq r(k-1,\ell)+r(k,\ell-1)$ for all $k,\ell\geq 2$.\\
                        \textbf{Answer}: Let $G=(V,E)$ be an arbitrary graph with $\abs{V}=r(k-1,\ell)+r(k,\ell-1)$, we want to show that it satisfies either $\omega(G)\geq k$ or $\alpha(G)\geq\ell$. Choose $u\in V$ to be an arbitrary vertex, then we can devide the rest of the vertices into two sets $A$ and $B$ depending on whether or not they are connected to $u$, respectively.\\
                        By construction of $A$ and $B$ we have $\abs{V}-1=\abs{A}+\abs{B}\implies r(k-1,\ell)+r(k,\ell-1)-1=\abs{A}+\abs{B}$; by the pigeonhole priciple we must have $\abs{A}\geq r(k-1,\ell)$ or $\abs{B}\geq r(k,\ell-1)$. Note that if $A$ contains $\ell$ independent vertices or if $B$ contains a complete subgraph of $k$ vertices, $G$ automatically satisfies either $\omega(G)\geq k$ or $\alpha(G)\geq\ell$, so we will examine the two following cases where the previous statement is not true.\\
                        If $\abs{A}\geq r(k-1,\ell)$, then $A$ is contains a complete subgraph with $k-1$ vertices. Since all those vertice are also connected to $u$, $G$ has a complete subgraph with $k$ vertices. Otherwise, if $\abs{B}\geq r(k,\ell-1)$, then $B$ contains $\ell-1$ independent vertices. Since all those vertices are not connected to $u$, $G$ has $\ell$ independent vertices.\\
                        Therefore $G$ satisfies either $\omega(G)\geq k$ or $\alpha(G)\geq\ell$. By definition of Ramsey's number, $r(k,l)$ is the minimum number of vertices of such a graph, so by definition of minimum we have $r(k,l)\leq\abs{V(G)}=r(k-1,\ell)+r(k,\ell-1)$.
                  \item [(c)] Use parts (a) and (b) above to obtain an upper bound for $r(5,5)$.\\
                        \textbf{Answer}: By part (a) we have $r(k,2)=k$ and by symmetry we also have $r(2,\ell)=\ell$. Then using part (b), We have \\
                        $r(5,5)$\\$\leq r(4,5)+r(5,4)$\\$\leq (r(3,5)+r(4,4))+(r(4,4)+r(5,3))$\\$=r(3,5)+2r(4,4)+r(5,3)$\\$\leq r(2,5)+r(3,4)+2(r(3,4)+r(4,3))+r(4,3)+r(5,2)$\\$=r(2,5)+3r(3,4)+3r(4,3)+r(5,2)$\\$\leq r(2,5)+3(r(2,4)+r(3,3))+3(r(3,3)+r(4,2))+r(5,2)$\\$=r(2,5)+3r(2,4)+6r(3,3)+3r(4,2)+r(5,2)$\\$\leq r(2,5)+3r(2,4)+6r(2,3)+6r(3,2)+3r(4,2)+r(5,2)$\\$=5+3\cdot 4+6\cdot 3+6\cdot 3+3\cdot 4+5$\\$=70$.
            \end{itemize}
\end{itemize}

\newpage
\begin{itemize}
      \item [Q4]
            \begin{itemize}
                  \item [(a)] Show that for $n=3$ and any $p\in[0,1]$, \[\mathlarger{\sum_{G\in\mathcal{G}_{3,p}}}P(G)=1\]
                        \textbf{Answer}: For $n=3$, we have $2^{\binom{3}{2}}=8$ and $0\leq\abs{E(G)}\leq 3$. We can examine the graphs based on $\abs{E(G)}$ as follows:
                        \begin{itemize}
                              \item [-] $\abs{E(G)}=0$: $P(G)=(1-p)^3$, we have $\binom{3}{0}=1$ such graph.
                              \item [-] $\abs{E(G)}=1$: $P(G)=p(1-p)^2$, we have $\binom{3}{1}=3$ such graphs.
                              \item [-] $\abs{E(G)}=2$: $P(G)=p^2(1-p)$, we have $\binom{3}{2}=3$ such graphs.
                              \item [-] $\abs{E(G)}=3$: $P(G)=p^3$, we have $\binom{3}{3}=1$ such graph.
                        \end{itemize}
                        Therefore we have $\mathlarger{\sum_{G\in\mathcal{G}_{3,p}}}P(G)=(1-p)^3+3p(1-p)^2+3p^2(1-p)+p^3$, which by cube of sum gives us $(p+(1-p))^3=1$.
                  \item [(b)] Prove that the events ``the graph $G$ is connected'' and ``the graph $G$ is bipartite'' are not independent in $\mathcal{G}_{3,p}$ unless $p=0$ or $p=1$.\\
                        \textbf{Answer}: For $p=0$, we have $P(G)=1$ for $\abs{E(G)}=0$ and $P(G)=0$ otherwise, i.e. $G$ has to be three independent vertices. Similarly, for $p=1$, we have $P(G)=1$ for $\abs{E(G)}=3$ and $P(G)=0$ otherwise, i.e. $G$ has to be connected. Therefore the events are independent in $\mathcal{G}_{3,p}$ for $p=0$ and $p=1$.\\
                        For other values of $p$, note that for $n=3$, $G$ is connected for $\abs{E(G)}\geq 2$ and is bipartite for $\abs{E(G)}\leq 2$. Since these two conditions overlap and $P(G)\neq 0$ for $0<p<1$, i.e. it is possible to have different graphs, we can conclude that the two events are not independent.
                  \item [(c)] If $p=0.1$, what is the probability that a random graph on the vertex set $\{1,2,3\}$, will not contain the edge $\{1,2\}$?\\
                        \textbf{Answer}: Since the probability of $G$ containing $\{1,2\}$ is $0.1$, the probability of $G$ not containing $\{1,2\}$ is $1-0.1=0.9$.
                  \item [(d)] Let $f$ be the random variable which assigns to a graph its number of edges. If $p=0.1$, what is the expected number of edges of a random graph on the vertex set $\{1,2,3\}$?\\
                        \textbf{Answer}: Since we can have up to 3 edges and each edge has $p=0.1$, the expected number of edges is $3\cdot 0.1=0.3$.
            \end{itemize}
\end{itemize}

\newpage
\begin{itemize}
      \item [Q5]
            \begin{itemize}
                  \item [(a)] Interpret this setup as a block design $t-(v,k,\lambda)$. What are the values of $t$, $v$, $k$, and $\lambda$?\\
                  \textbf{Answer}: We have $t=2$, $v=16$, $k=4$ and $\lambda=2$.
                  \item [(b)] How many weeks are in the course?\\
                  \textbf{Answer}: By integrality conditions, we have $\lambda\dfrac{v(v-1)}{k(k-1)}=2\cdot\dfrac{16\cdot 15}{4\cdot 3}=40$ blocks, so there are 40 weeks in the course.
                  \item [(c)] How many presentations does each student give?\\
                  \textbf{Answer}: Again by integrality conditions, we have $\lambda\dfrac{(v-1)}{k-1}=2\cdot\dfrac{15}{3}=10$ repetitions, so each student gives 10 presentations.
            \end{itemize}
\end{itemize}

\newpage
\begin{itemize}
      \item [Q6]
            \begin{itemize}
                  \item [(a)] What is the dimension of the cycle space $\mathcal{E}$ of $K_5$?\\
                        \textbf{Answer:} By Theorem 13.4.3, we have $\dim(\mathcal{E})=\abs{E}-\abs{V}+k=10-5+1=6$.
                  \item [(b)] Give a basis for $\mathcal{E}$ with respect to the spanning tree whose edge set is \[E'=\{\{1,2\},\{1,3\},\{3,5\},\{4,5\}\}\]
                        \textbf{Answer}: Let $B=\{b_1,b_2,b_3,b_4,b_5,b_6\}$ be a basis for $\mathcal{E}$ with\\
                        $b_1=\{\{1,2\},\{1,3\},\{2,3\}\}$\\
                        $b_2=\{\{1,3\},\{1,5\},\{3,5\}\}$\\
                        $b_3=\{\{3,4\},\{3,5\},\{4,5\}\}$\\
                        $b_4=\{\{1,2\},\{1,3\},\{2,5\},\{3,5\}\}$\\
                        $b_5=\{\{1,3\},\{1,4\},\{3,5\},\{4,5\}\}$\\
                        $b_6=\{\{1,2\},\{1,3\},\{2,4\},\{3,5\},\{4,5\}\}$.
                  \item [(c)] How many even sets are in $K_5$?\\
                        \textbf{Answer}: By Corollary 13.4.4, $K_5$ has $2^{\abs{E}-\abs{V}+k}=2^{6}=64$ even sets.
                  \item [(d)] How many cycles are in $K_5$? Give an example of a nonempty even set of $K_5$ which is not a cycle and is not all of $K_5$.\\
                        \textbf{Answer}: By formula 13.2, we have $\abs{\mathcal{K}_{K_n}}=\mathlarger{\sum_{k=3}^n\binom{n}{k}}\cdot\dfrac{(k-1)!}{2}=\mathlarger{\sum_{k=3}^5\binom{5}{k}}\cdot\dfrac{(k-1)!}{2}=\mathlarger{\binom{5}{3}}\cdot\dfrac{2!}{2}+\mathlarger{\binom{5}{4}}\cdot\dfrac{3!}{2}+\mathlarger{\binom{5}{5}}\cdot\dfrac{4!}{2}=10+15+12=37$.\\
                        An example of such an even set is $\{\{1,2\},\{1,3\},\{1,4\},\{1,5\},\{2,5\},\{3,4\}\}$. The vertices are all even as vertex 1 is degree 4 and the others are degree 2; in addition, the set is not a cycle since a cycle is degree 2 everywhere.
            \end{itemize}
\end{itemize}

\end{document}
\documentclass{article}
\usepackage[margin=1in]{geometry}
\usepackage{enumitem}
\usepackage{setspace}
\usepackage{amsmath}
\usepackage{amssymb}
\usepackage{physics}
\usepackage{fancyhdr}
\pagestyle{fancy}
\lhead{Jiaping Zeng}
\rhead{Disc. 1A, 10AM, Lionel Levine}

\title{CS 180 Homework 2}
\date{10/18/2020}
\author{Jiaping Zeng}

\begin{document}
\setstretch{1.35}

\begin{itemize}
    \item [2.8b]
          \textbf{Algorithm}: From the previous homework, we found that the optimal starting position for 2 eggs, $n$ floors is $x=\lceil\frac{-1+\sqrt{1+8n}}{2}\rceil$ where the $n$ is the number of floors. We will start from there:
          \begin{itemize}
              \item [1.] Set step size $x=\lceil\frac{-1+\sqrt{1+8n}}{2}\rceil$.
              \item [2.] While jar doesn't break, set new step size $x=x-1$, step up and drop the jar.
              \item [3.] Jar is now broken; set new floor count $n=x$ and return to last step.
              \item [4.] Repeat steps 1-3 until there is 1 jar remaining, in which case set step size $x=1$ and step up until the last jar breaks.
          \end{itemize}
          \textbf{Proof of correctness}: Since $f_k$ is defined recursively as $f_k(n)=\frac{-1+\sqrt{1+8f_{k-1}(n)}}{2}$, by substitution we have $\lim_{n\rightarrow\infty}\frac{f_k(n)}{f_{k-1}(n)}=\lim_{n\rightarrow\infty}\frac{-1+\sqrt{1+8f_{k-1}(n)}}{2f_{k-1}(n)}=0$. Therefore each function does indeed grow asymptotically slower than the previous one.\\
          \textbf{Complexity}: $O(k\log n)$, since we need to traverse through $k$ jars and the trials of each jar is done in $O(\log n)$.
    \item [3.5]
          \textbf{Answer}: Proof by induction;\\
          Base case: Take a binary tree with exactly one node, then that node is a leave and there is no node with two children. Therefore the number of nodes with two children is exactly one less than the number of leaves for a binary tree with a single node.\\
          Inductive step: Suppose that any binary tree with $n$ leaves has $n-1$ nodes with two children, we want to show that any binary tree with $n+1$ leaves has exactly $n$ nodes with two children. We can think of this as adding a node to the binary tree with $n$ leaves. There are two possible scenarios from here:
          \begin{itemize}
              \item [-] Add the new node to a leaf node: the new node becomes a leaf node, while its parent is no longer a leaf node. So the number of leaves remain at $n$ and the number of nodes with two children remain at $n-1$. So this scenario actually does not create a new leaf node, but we can observe that the rule still applies.
              \item [-] Add the new node to a node with 1 child: the new node becomes a leaf node, while its parent becomes a node with two children. Then the number of leaves become $n+1$ and the number of nodes with two children becomes $n$.
          \end{itemize}
          Therefore the number of nodes with two children is exactly one less than the number of leaves by mathematical induction.
    \item [3.7]
          \textbf{Answer}: True. Proof by contradiction:\\
          Suppose that there exists a disconnected graph $G$, with $n$ nodes, $n$ even, where every node of $G$ has degree at least $\frac{n}{2}$. Since $G$ is disconnected, there must be at least two components. However, since every node has degree at least $\frac{n}{2}$, each component must have at least $\frac{n}{2}+1$ nodes. But this implies $G$ has at least $2(\frac{n}{2}+1)=n+2$ nodes, which contradicts with our assumption that $G$ has $n$ nodes, i.e. such disconnected $G$ does not exist. Therefore by contradiction, if every node of $G$ has degree at least $\frac{n}{2}$, then $G$ is connected.
    \item [3.10]
          \textbf{Algorithm}:
          \begin{itemize}
              \item [1.] Start at vertex $w$, traverse through its adjacent vertices. Store these as a list $l$.
              \item [2.] If $w$ is in $l$, return 1.
              \item [3.] Traverse through the unvisited adjacent vertices of vertices in $l$, store them as the new $l$.
              \item [4.] If $w$ is in $l$, return the number of times it appears.
              \item [5.] Repeat steps 3-4 until $w$ is found in $l$.
          \end{itemize}
          \textbf{Proof of correctness}: We will first show that the algorithm does indeed find the shortest path by contradiction. Suppose that the algorithm finds paths of length $k$ from $v$ to $w$, but there exists at least one path of length $j$ such that $j<k$. Since each iteration of the algorithm represents moving one length away, paths of length $k$ are found in iteration $k$ and paths of length $j$ are found in iteration $j$. However, $j<k$ implies that we have already checked all paths of length $j$ in a previous iteration, which means the algorithm would have returned paths of length $j$ instead. Therefore such $j$ does not exist and the algorithm does find the shortest paths.\\
          Now we will show that the algorithm returns the correct number of shortest paths. Since by construction $l_k$ contains all vertices length $k$ away from $u$, any path from $u$ to $w$ would be contained in $l_k$. Therefore it is not possible for the algorithm to miss paths of length $k$.\\
          \textbf{Complexity}: $O(m+n)$ as the worst case scenario is traversing through every edge and vertex, i.e. $m+n$.
    \item [P1] Suppose that you are given an algorithm as a blackbox. You cannot see how it is designed. The blackbox has the following properties: if you input any sequence of real numbers, and an interger $k$, the algorithm will answer YES or NO indicating whether there is a subset of the numbers whose sum is exactly $k$. Show how to use this blackbox to find the subset whose sum is $k$, if it exists. You should use the blackbox O($n$) times, where $n$ is the size of the input sequence.\\
          \textbf{Algorithm}: Let $B(\cdot)$ denote the boolean blackbox algorithm and $arr$ denote the unsorted sequence of real numbers.
          \begin{itemize}
              \item [1.] If $B(arr, k)$ returns NO, exit program as such subset does not exist.
              \item [2.] Set $subset=[\;]$.
              \item [3.] Traverse through the sequence and check $B(arr,k-arr[i])$, where $arr[i]$ is the current element:
                    \begin{itemize}
                        \item [-] If $B(arr,k-arr[i])$ returns YES:
                              \begin{itemize}
                                  \item [-] Append $arr[i]$ to $subset$.
                                  \item [-] Set $k=k-arr[i]$.
                                  \item [-] if $k==0$, exit loop.
                              \end{itemize}
                        \item [-] Else continue.
                    \end{itemize}
              \item [4.] Return $subset$.
          \end{itemize}
          \textbf{Proof of correctness}: Let $S$ be the desired subset. We will show that the algorithm will always return $S$ by induction.\\
          Base case: When the desired set has only one element, i.e. $\abs{S}=1$; let $s_1$ be the only element of $S$ (note that $k=s_1$), then we have the following two possible scenarios:
          \begin{itemize}
              \item [1.] $s_1$ is not present in the sequence: this is handled by step 1; $B(arr,k)$ would return NO and exit the program as it is not possible to construct a subset with sum $k$ from the sequence.
              \item [2.] $s_1$ is in the sequence: since step 3 traverses through every element in the sequence, it will eventually traverse to $s_1$. Since $k=s_1$, $B(arr,k-s_1)$ will always evaluate to true as it is always possible to construct a subset of sum 0 by selecting the empty set.
          \end{itemize}
          Inductive step: Suppose that given $k_n=s_1+\ldots+s_n$, the algorithm successfully returns $S_n=\{s_1,\ldots,s_n\}$. We want to show that given $k_{n+1}=k_n+s_{n+1}$, the algorithm will return $S_{n+1}=S_n\cup\{s_{n+1}\}$. We can show that such $s_{n+1}$ always exist and will be selected by the algorithm by examining the following scenarios:
          \begin{itemize}
              \item [1.] Such $s_{n+1}$ does not exist: in this scenario, $B(arr,k_{n+1})$ would return NO and the program would exit after step 1.
              \item [2.] Such $s_{n+1}$ exists: suppose $s_{n+1}$ appears in the sequence after each $s_i\in S_n$ has already been traversed, which we can guarantee upon renumbering. Then $B(arr,k_{n+1}-s_{n+1})$ is equivalent to $B(arr,k_n)$ which is assumed true by inductive hypothesis. Then the algorithm will construct and return $S_{n+1}$ by appending $s_{n+1}$ to $S$, as desired.
          \end{itemize}
          Therefore the algorithm will always return a complete and correct result by mathematical induction.
          \textbf{Complexity}: $O(n)$, assuming $B(\cdot)$ is $O(1)$, since the algorithm traverses through the sequence only once.
    \item [P2] An array of $n$ elements contains all but one of the integers from $1$ to $n+1$.
          \begin{itemize}
              \item [(a)] Give the best algorithm you can for determining which number is missing if the array is sorted, and analyze its asymptotic worst-case running time.\\
                    \textbf{Algorithm}:
                    \begin{itemize}
                        \item [1.] Set lower search (inclusive) bound $lower=0$ and upper search (exclusive) bound $upper=n$.
                        \item [2.] While $upper-lower>1$, visit the element at index $i=\lfloor\frac{lower+upper}{2}\rfloor$ (assuming 0-based indexing) and check its value $arr[i]$:
                              \begin{itemize}
                                  \item [-] If $arr[i]=i+1$, the missing element is in the second half of the current search interval. Set $lower=i$.
                                  \item [-] If $arr[i]=i+2$, the missing element is in the first half of the current search interval. Set $upper=i$.
                              \end{itemize}
                        \item [3.] The search area is now exactly one number (with $lower$ as its index), meaning that we have found the neighbor of the missing number. Check which side the missing number is on:
                              \begin{itemize}
                                  \item [-] If $arr[lower]=lower+1$, return $arr[lower]+1$ as the missing number.
                                  \item [-] If $arr[lower]=lower+2$, return $arr[lower]-1$ as the missing number.
                              \end{itemize}
                    \end{itemize}
                    \textbf{Proof of correctness}: By induction on the size of the array.\\
                    Base case: $n=1$, then our array contains 1 element ranging from 1 to 2, i.e. either $arr=[1]$ or $arr=[2]$. We can examine the two possible scenarios separately:
                    \begin{itemize}
                        \item [-] $arr=[1]$: we have $lower=0$ and $upper=1$, skipping step 2 as $upper-lower=1\ngtr 1$, then since $arr[lower]=arr[0]=1=lower+1$, return $2$ as the missing number, which is correct.
                        \item [-] $arr=[2]$: we have $lower=0$ and $upper=1$, skipping step 2 as $upper-lower=1\ngtr 1$, then since $arr[lower]=arr[0]=2=lower+2$, return $1$ as the missing number, which is correct.
                    \end{itemize}
                    Inductive step: Assume that the algorithm successfully finds the missing number for an array of size up to $n$. We want to show that it will also work for an array of size $n+1$. There are two possible scenarios here:
                    \begin{itemize}
                        \item [-] $n+1$ is even: the array is halved into two search areas of length $\frac{n+1}{2}$. Since $\frac{n+1}{2}\leq n$, we know that the algorithm works by inductive hypothesis.
                        \item [-] $n+1$ is odd: the array is halved into two search areas of sizes $\frac{n}{2}$ and $\frac{n}{2}+1$. Since $\frac{n}{2}<n$ and $\frac{n}{2}+1<n$, we know that the algorithm works by inductive hypothesis.
                    \end{itemize}
                    Therefore the algorithm will always return the missing number by induction.\\
                    \textbf{Complexity}: $O(\log n)$; the algorithm halves the search area until the search area is size 1, taking $\log_2 n$ iterations.
              \item [(b)] Give the best algorithm you can for determining which number is missing if the array is not sorted, ana analyze its asymptotic worst-case running time.\\
                    \textbf{Algorithm}:
                    \begin{itemize}
                        \item [1.] Set $sum=0$.
                        \item [2.] Traverse through the array and add each element to the sum, i.e. $sum=sum+arr[i]$.
                        \item [3.] Return $\frac{1}{2}(n+1)(n+2)-sum$ as the missing number.
                    \end{itemize}
                    \textbf{Proof of correctness}: The accumulated $sum$ includes every integer from $1$ to $n+1$, excluding the missing number. Since we know that the sum of the first $n+1$ elements is $\frac{1}{2}(n+1)(n+2)$, we can simply subtract $f$.\\
                    \textbf{Complexity}: $O(n)$, since it requires traversing through an array of $n$ elements.
          \end{itemize}
\end{itemize}
\end{document}
\documentclass{article}
\usepackage[margin=1in]{geometry}
\usepackage{enumitem}
\usepackage{setspace}
\usepackage{amsmath}
\usepackage{amssymb}
\usepackage{physics}
\usepackage{fancyhdr}
\pagestyle{fancy}
\lhead{Jiaping Zeng}
\rhead{Disc. 1A, 10AM, Lionel Levine}

\title{CS 180 Homework 1}
\date{10/12/2020}
\author{Jiaping Zeng}

\begin{document}
\setstretch{1.35}

\begin{itemize}
      \item [1.3]
            \textbf{Answer:} Let $a_1$ and $a_3$ be shows of network $\mathcal{A}$, and shows $b_2$ and $b_4$ be shows of network $\mathcal{B}$, where the subscript of each show denotes the ratings of the show. Then we have two possiblities for $(S,T)$, the pair of schedules: $((a_1,b_2),(a_3,b_4))$ and $((a_1,b_4),(a_3,b_2))$. In the first case, network $\mathcal{B}$ gets both slots; however, network $\mathcal{A}$ can simply reverse the order of its shows and win the first time slot. In the second case, both networks get one slot, but network $B$ can reverse the order of its show to win both slots. Therefore there is no stable pair of schedules in this case.
      \item [1.5]
            \begin{itemize}
                  \item [(a)] \textbf{Answer:} There always exists a perfect matching with no strong instability. We can update the G-S algorithm by specifying that if a man $m$ is indifferent between women $w_i$ and $w_j$, select $w_i$ over $w_j$ if $i<j$ and $w_j$ over $w_i$ otherwise. Similarly, a woman will also select between equally-preferred men $m_i$ over $m_j$ if $i<j$ and $m_j$ over $m_i$ otherwise.\\Since the G-S algorithm always returns a stable matching (no instability), there cannot be any strong instability in this updated G-S algorithm.
                  \item [(b)] \textbf{Answer:} There does not always exist matching with no weak instability. Let $m_1$ and $m_2$ be indifferent about $w_1$ and $w_2$, and both $w_1$ and $w_2$ prefer $m_1$ over $m_2$. If we have matches $(m_1,w_1),(m_2,w_2)$, $w_2$ prefers $m_1$ and $m_1$ is indifferent. If we have matches $(m_1,w_2),(m_2,w_1)$, $w_1$ prefers $m_1$ who is again indifferent. Therefore there is no matching with no weak instability in this case.
            \end{itemize}
      \item [1.7]
            \textbf{Answer:} We can solve this problem by translating it into the stable matching problem, then applying the G-S algorithm. We can construct the preference list for each input wire by ranking the output wires from upstream to downstream; similarly, we can construct the preference list for each output wire by ranking the input wires from downstream to upstream. Also note that switching in this problem is analogous to marriage in the stable matching problem. Then we can apply the G-S algorithm, guaranteeing a stable matching, which in this case is a valid switching. We can prove this as follows:\\
            Suppose the stable matching generated by the G-S algorithm is not a valid switching, then it would cause two data streams to cross. Let $(i_m,o_m)$ and $(i_n,o_n) $ be the crossed streams, where $i_m,i_n$ are input wires and $o_m,o_n$ are output wires. Since the streams are crossed, $i_m$ prefers $o_n$ over $o_m$ which contradicts our assumption of a stable match. Therefore a stable match is a valid switching.
      \item [2.2]
            \textbf{Answer:} $10^{10}$ ops/sec = $6\cdot 10^{11}$ ops/min = $3.6\cdot 10^{13}$ ops/hr. Then the problem is equivalent to finding the largest $n\in\mathbb{Z}$ (rounded down) that satisfies $f(n)=3.6\cdot 10^{13}$, where $f(n)$ is the running time of the algorithms.
            \begin{itemize}
                  \item [(a)] $n^2$: $n=\sqrt{3.6\cdot 10^{13}}=\boxed{6\cdot 10^6}$
                  \item [(b)] $n^3$: $n=\sqrt[3]{3.6\cdot 10^{13}}\approx\boxed{33019}$
                  \item [(c)] $100n^2$: $n=\sqrt{3.6\cdot 10^{11}}=\boxed{6\cdot 10^5}$
                  \item [(d)] $n\log n$: Solving numerically, $n\approx\boxed{1.29\cdot 10^{12}}$
                  \item [(e)] $2^n$: $n=\log_2 3.6\cdot 10^{13}\approx\boxed{45}$
                  \item [(f)] $2^{2^n}$: $n\approx\boxed{5}$
            \end{itemize}
      \item [P1]
            \begin{itemize}
                  \item [(a)] Prove (by induction) that sum of the first $n$ integers $1+2+\ldots+n$ is $\frac{n(n+1)}{2}$.\\
                        \textbf{Proof:} By induction; \\Base case ($n=1$): $\sum_{k=1}^1 k=1=\frac{n(n+1)}{2}$.\\Inductive step: Assume that $\sum_{k=1}^n k=\frac{n(n+1)}{2}$ holds, we want to show that $\sum_{k=1}^{n+1} k=\frac{(n+1)(n+2)}{2}$. By substitution we have \\$\sum_{k=1}^{n+1} k$\\$=(n+1)+\sum_{k=1}^n k$\\$=(n+1)+\frac{n(n+1)}{2}$\\$=\frac{2(n+1)+n(n+1)}{2}$\\$=\frac{(n+1)(n+2)}{2}$.\\Therefore $\sum_{k=1}^n k=\frac{n(n+1)}{2}$ by mathematical induction.
                  \item [(b)] What is $1^2+2^2+\ldots+n^2$? Prove your answer by induction.\\
                        \textbf{Answer:} $\sum_{k=1}^n k^2=\frac{n(n+1)(2n+1)}{6}$\\\textbf{Proof:} By induction;\\Base case ($n=1$): $\sum_{k=1}^1 k=1=\frac{n(n+1)(2n+1)}{6}$.\\Inductive step: Assume that $\sum_{k=1}^n k^2=\frac{n(n+1)(2n+1)}{6}$ holds, we want to show that $\sum_{k=1}^{n+1} k^2=\frac{(n+1)(n+2)(2n+3)}{6}$. By substitution we have \\$\sum_{k=1}^{n+1} k^2$\\$=(n+1)^2+\frac{n(n+1)(2n+1)}{6}$\\$=\frac{6(n+1)^2+n(n+1)(2n+1)}{6}$\\$=\frac{(n+1)[6(n+1)+n(2n+1)]}{6}$\\$=\frac{(n+1)(6n+6+2n^2+n)}{6}$\\$=\frac{(n+1)(n+2)(2n+3)}{6}$.\\Therefore $\sum_{k=1}^n k^2=\frac{n(n+1)(2n+1)}{6}$ by mathematical induction.
            \end{itemize}
      \item [P2] How many tries do you need (in the worst case) in the two egg problem when there are 200 steps? What about $n$ steps?\\
            \textbf{Answer:} Suppose we have the best strategy, let $x$ be the minimum number of drops in the worst case. Then the drop points would be floors $x,x+(x-1),x+(x-1)+(x-2),\ldots$, covering a total of $x+(x-1)+(x-2)+\ldots+2+1=\frac{x(x+1)}{2}$ floors.\\
            For 200 floors, we have $\lceil\frac{x(x+1)}{2}\rceil=200\implies x=\lceil 19.51\rceil=\boxed{20}$.\\
            Similarly, for $n$ floors, we have $\lceil\frac{x(x+1)}{2}\rceil=n\implies\lceil x^2+x-2n=0\rceil\implies x=\boxed{\lceil\frac{-1+\sqrt{1+8n}}{2}\rceil}$
\end{itemize}
\end{document}
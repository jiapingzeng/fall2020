\documentclass{article}
\usepackage[margin=1in]{geometry}
\usepackage{enumitem}
\usepackage{setspace}
\usepackage{amsmath}
\usepackage{amssymb}
\usepackage{physics}

\title{CS 180 Homework 6}
\date{11/29/2020}
\author{Jiaping Zeng}

\begin{document}
\setstretch{1.35}
\maketitle

\begin{itemize}
    \item [6.20]
          \textbf{Algorithm}:
          \begin{itemize}
              \item [1.] Let $G(c,h)$ be the maximum total grade for $c$ courses using $h$ hours. Set $G(0,h)=0$ and $G(c,0)=\sum_{i=0}^c f_i(0)$.
              \item [2.] For each course $i$, do the following:
                    \begin{itemize}
                        \item [-] For $0\leq h\leq H$, set $G(i,h)=\max_{0\leq t\leq h}\{f_i(t)+G(i-1,h-t)\}$.
                        \item [-] Store the corresponding $t$ that maximizes $G(i,h)$.
                    \end{itemize}
              \item [3.] $G(n,H)$ is our maximum total grade, backtrack to retrieve the hours spent for each course.
          \end{itemize}
          \textbf{Proof of correctness}: By induction on $n$, the number of courses.\\
          Base case: $n=0$, since $G(0,h)$ is initialized to 0 in step 1, our algorithm returns the correct result.\\
          Inductive step: Suppose our algorithm is correct for $n$ courses, we want to show that it is also correct for $n+1$ courses. Note that when calculating $G(n+1,h)$, we are finding the maximum of $G(n,h-t)$ values for various $t$. By inductive hypothesis, these $G(n,h-t)$ values are correct. So our algorithm is correct for $n+1$ courses.\\
          Therefore our algorithm is correct by induction.\\
          \textbf{Time complexity}: $O(nH^2)$; looping through each course is $O(n)$, then looping through $H$ is $O(H)$ and finding the maximum grade each time is $O(H)$. Therefore our main loop is $O(nH^2)$ and backtracking is $O(n)$, so overall our algorithm is $O(nH^2)$.
    \item [6.21]
          \textbf{Algorithm}:
          \begin{itemize}
              \item [1.] Let $R(i,j)$ denote the maximum return of a buy $b$ and sell $s$ between days $i$ and $j$, i.e. $R(i,j)=\max p(s)-p(b)$ for $i\leq b\leq s\leq j$.
              \item [2.] For each interval between days $i$ and $j$, set $R(i,j)=\max\{p(i)-p(j),R(i,j-1),R(i+1,j)\}$.
              \item [3.] Let $P(m,d)$ be the maximum profit of an $m$-shot strategy over $d$ days. Set $P(0,\cdot)=0$. In addition set $P(m,d)=-\infty$ if $2m>d$ since it is not possible to execute such strategies.
              \item [4.] For each $1\leq m\leq k$ and $1\leq d\leq n$, set $P(m,d)=\max_{1\leq i<j\leq d}R(i,j)+P(m-1,i-1)$. Store the days $i$ and $j$ that corresponds to the maximum profit strategy.
              \item [5.] $P(k,n)$ is the maximum profit of the $k$-shot strategy, backtrack using stored intervals to retrieve the buy/sell dates.
          \end{itemize}
          \textbf{Proof of correctness}: By induction on $n$, the number of days.\\
          Base case: $n=2$, then we return a strategy with $-\infty$ profit for $k\geq 2$ as such strategies are not possible; for $k=0$ we return a strategy with no profit since there is no transaction; for $k=1$ we return a strategy that buys on day 1 and sells on day 2 as expected.\\
          Inductive step: Suppose our algorithm works a $k$-shot strategy with $n$ days, we want to show that it will also work for one with $n+1$ days. Note that in the calculation of $P(k,n+1)$, we have $1\leq i\leq d\leq n+1\implies i-1\leq n$, so $P(m-1,i-1)$ would cover only strategies over $n$ days or less. By inductive hypothesis we will always have the correct value for those, so we will also have the correct value for $P(k,n+1)$.\\
          Therefore our algorithm works by induction.\\
          \textbf{Time complexity}: $O(kn^2)$; constructing $R(i,j)$ takes $O(n^2)$. Calculating $P(m,d)$ takes $O(kn)$ for the outer loop and $O(n)$ per iteration, so it is $O(kn^2)$. Therefore overall our algorithm is $O(kn^2)$.
    \item [6.24]
          \textbf{Algorithm}: Let $A$ be the total number of party A votes and $a_i$ denote the number of party A votes in precinct $P_i$. Note that party A would need $\frac{nm}{4}+1$ votes in each district to win.
          \begin{itemize}
              \item [1.] Let $G(p,q,v)$ return true there is a set of $p$ precincts out of the first $q$ precincts that contains exactly $v$ party A votes. Set $G(0,\cdot,0)=True$.
              \item [2.] For each precinct $1\leq p\leq\frac{n}{2}$, then for each precinct $1\leq q\leq n$, do the following:
                    \begin{itemize}
                        \item [-] For $1\leq v\leq A-\frac{nm}{4}-1$, set $G(p,q,v)=True$ if $p=q=1$ and $v=a_1$, or if $G(p-1,q-1,v-a_p)=True$, or if $G(p,q-1,v)=True$.
                        \item [-] Else set $G(p,q,v)=False$.
                    \end{itemize}
              \item [3.] Return true if any $G(\frac{n}{2},n,v)$ for $\frac{nm}{4}+1\leq v\leq A-\frac{nm}{4}-1$ is true.
          \end{itemize}
          \textbf{Proof of correctness}: By induction on $n$, the number of precincts.\\
          Base case: we have 1 precinct, then our algorithm checks if we have $v=a_1$ and returns correspondingly, as expected. (This does not make much sense in terms of gerrymandering since 1 precinct cannot be split up into districts, however it is a functional base case.)\\
          Inductive step: Suppose our algorithm works for $n$ precincts, we want to show that it will also work for $n+1$ precincts. Note for the $n+1$ iteration, our $q$ in $G(p,q,v)$ is at most $n$ which works by inductive hypothesis. Therefore our algorithm must also return the correct result for $n+1$ precincts.\\
          Therefore our algorithm works by induction.\\
          \textbf{Time complexity}: $O(n^3m)$; our main loop is $O(n^3m)$ and checking entries of $G$ takes $O(nm)$, so overall the algorithm is $O(n^3m)$.
    \item [7.8]
          \begin{itemize}
              \item [(a)]
                    \textbf{Algorithm}:
                    \begin{itemize}
                        \item [1.] We set up a four-levels graph as follows:
                              \begin{itemize}
                                  \item [-] Level 1: source node
                                  \item [-] Level 2: four supply nodes, one for each blood type
                                  \item [-] Level 3: four demand nodes, one for each blood type
                                  \item [-] Level 4: sink node
                              \end{itemize}
                        \item [2.] Connect the source node to each supply node with capacity set to the supply; similarly, connect the sink node to each demand node with capacity set to the demand.
                        \item [3.] Between the supply nodes and demand nodes, construct an edge only if the demand type can resource from the supply type. Set capacity to the demand.
                        \item [4.] Run Ford-Fulkerson to find the max-flow. Return true if our max-flow saturates demand edges to the sink and false otherwise.
                    \end{itemize}
                    \textbf{Proof of correctness}: By contradiction; suppose our algorithm is incorrect, then it would either return true for insufficient supplies or false for sufficient supplies, both of which contradicts with that Ford-Fulkerson is correct. Therefore our algorithm also return the correct result by contradiction.\\
                    \textbf{Time complexity}: $O(em)$ since Ford-Fulkerson is $O(em)$ where $e=\abs{E}$ and $m$ is the max-flow.
              \item [(b)] There is not enough supply since we have 86 units of supplies of types O and A with 87 units of demand and they cannot receive from other types.\\
                    We can first use all supplies of AB and O to fulfill their demands, afterwards the remaining cannot be used for other types. So then we have 50 units of supply for type O with 45 units of demand; similarly we have 36 units of supply for type A with 42 units of demand. We can first fulfill the type O demand with 5 units leftover, which we can use to help with type A demand, leaving us with $42-36-5=1$ person who does not receive blood.
          \end{itemize}
    \item [7.12]
          \textbf{Algorithm}:
          \begin{itemize}
              \item [1.] Run Ford-Fulkerson algorithm; from the residual graph, find the reachable vertices. Store all edges from a reachable vertex to a nonreachable vertex as the min-cut edges.
              \item [2.] If we have more than $k$ edges, select any $k$ edges from them. Return such edges.
              \item [3.] Else if we have less than $k$ edges, add any remaining edges until we have $k$ edges. Return the edges.
          \end{itemize}
          \textbf{Proof of correctness}: By contradiction. Suppose our algorithm does not give us the desired edges, i.e. the edges are not in the min-cut. However, since Ford-Fulkerson is correct, we will always have edges in the min-cut. Therefore our algorithm is correct by contradiction.\\
          \textbf{Time complexity}: $O(em)$ since Ford-Fulkerson is $O(em)$ where $e=\abs{E}$ and $m$ is the max-flow.
    \item [P6]
          \textbf{Algorithm}: We will assume that all elements in the input sequence are unique.
          \begin{itemize}
              \item [1.] Let $S_>(i)$ be the length of the longest alternating subsequence ending at index $i$ with the last element greater than the previous element; let $S_<(i)$ be the same but with the last element smaller than the previous element. Set $S_>(\cdot)=S_<(\cdot)=1$.
              \item [2.] For each element $x_i$ of the list, do the following:
                    \begin{itemize}
                        \item [-] For $0\leq j<i$, compare $x_i$ and $x_j$.
                        \item [-] If $x_i>x_j$, set $S_>(i)=\max\{S_>(i),S_<(j)+1\}$.
                        \item [-] Else if $x_j>x_i$, set $S_<(i)=\max\{S_<(i),S_>(j)+1\}$.
                    \end{itemize}
              \item [3.] $\max\{S_>(i),S_<(i)\}$ is the length of our longest alternating subsequence, backtrack to retrieve the subsequence and return it.
          \end{itemize}
          \textbf{Proof of correctness}: By induction on $n$, the length of the given sequence.\\
          Base case: $n=1$; the loop in step 2 will simply be skipped since there is no $j<i$. Then we return the original sequence as the longest alternating subsequence.\\
          Inductive step: Suppose our algorithm works for an input sequence of length $n$, we want to show that it will also work for an input sequence of length $n+1$. For the last element $x_{n+1}$, our algorithm will calculate either $\max\{S_>(n+1),S_<(j)+1\}$ or $\max\{S_<(n+1),S_>(j)+1\}$. $S_>(n+1)$ and $S_<(n+1)$ will be correct as we initialized them to 1 at the start, and we are only updating them respectively if values of $S_<(j)+1$ and $S_>(j)+1$ are greater. Note that since we set $j<i$, $j$ is at most $n$. Then by inductive hypothesis our $S_>(j)$ and $S_<(j)$ calculations will be correct.\\
          Therefore our algorithm is correct by induction.\\
          \textbf{Time complexity}: $O(n^2)$; looping through $x_j$ for every $x_i$ is $O(n^2)$, then backtracking to retrieve the subsequence is $O(n)$, so overall the algorithm is $O(n^2)$.
\end{itemize}
\end{document}